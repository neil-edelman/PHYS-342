\documentclass[twocolumn]{article}

% imports
\usepackage{times}      % font
\usepackage{graphicx}   % include graphics
\usepackage{fullpage}   % book margins -> std margins
\usepackage{amsmath}    % {align}
\usepackage{wrapfig}    % {wrapfigure}
\usepackage{moreverb}   % \verbatimtabinput

\bibliographystyle{plain}

% define "\fig"
\def\fig#1#2{\begin{figure}[!ht]\begin{center}
\includegraphics[width=0.32\textwidth]{#1.jpg}
\end{center}\caption{#2}\label{#1}\end{figure}}

% lists
\renewcommand{\labelenumi}{\arabic{enumi})}
\renewcommand{\labelenumii}{\alph{enumii})}

% Muhamed Razalee Yusoff -- 260263544,
% Neil Edelman -- 110121860,
% Daniel Smilowitz -- 260214511,
% Olivier Shelbaya -- 260279844

% info
\author{Muhamed Razalee Yusoff, Neil Edelman, Daniel Smilowitz, Olivier Shelbaya}
\title{Magnetic Simulation of Exploding Planet by Death Star}
\date{2009-02-15}

\begin{document}\maketitle

\begin{abstract}
The Gauss-Seidel iterative algorithm was used to simulate the magnetic
field on a constant grid. Each grid point has a possibly different
permeability, this allows magnetism to propagate through different
mediums. We used it to simulate a planet much like the Earth, represented
by a rotating molten magma core and a surrounding region of Olivine.
This planet is then hit by a positron beam from the Death Star which destroys it.
\end{abstract}

\section{Introduction}
The Death Star is the ultimate weapon of terror in the universe.
Modeling the Death Star as a sphere of iron, we set out to not only
map the interacting magnetic fields of Earth, but also that of the
Death Star as it orbits at a distance. Unlike the Earth, the Death
Star, given a value of $\mu_{Fe}$, does not generate a magnetic field,
so much as it is permeated by that of the Earth.

\section{Discussion}

The simulation starts off with no current, and an isotropic $\mu_{0}$
spread out over the volume. The magma is then generated with a circular
current simulating the core of Earth. Olivine is added to form the mantle
and this changes the magnetic field because $\mu \ne \mu_{0}$. The Death
Star appears, having $\mu_{Fe}$. Our simulation also involves a positron
beam which destroys all of human life. Evidently, this beam will generate
a magnetic field, as it constitutes a large current.

\section{Theory}

In general, the equation linking the magnetic vector potential,
$A(r)$, and the associated current is given by
Equation~\ref{eq-current}\cite{griffiths}.

\begin{equation}
\nabla \times (\frac{1}{\mu} \nabla \times A) = J
\label{eq-current}
\end{equation}

This equation was used to find the vector potential from the model
currents. Next, we need to model the magnetic field proper. This was
done by using Equation~\ref{eq-mag}\cite{griffiths}.

\begin{equation}
B = \nabla \times A
\label{eq-mag}
\end{equation}

In order to implement this, we need a grid system to attach a
vectorial value at each point of space in our given grid. We used
a rudimentary simple grid (without quadtrees.) Then we used the
Gauss-Siedel\cite{jackson} algorithm in order to evaluate the values
at each point in our three-space. In general, the Gauss-Siedel algorithm
can be expressed as in Equation~\ref{gauss}.

{\footnotesize \begin{equation}
x^{(k+1)}_i  = \frac{1}{a_{ii}}
\left(b_i - \sum_{j<i}a_{ij}x^{(k+1)}_j-
\sum_{j>i}a_{ij}x^{(k)}_j\right),\, i=1,2,\ldots,n.
\label{gauss}
\end{equation}}

A relaxation factor of $1.9$ was used for the computation. It was found
that this gave good convergence.

The simulated Earth-like planet was a core of iron with a $\mu$ of
$5000$\cite{taylor}. We found our model converged very slowly with such
an extreme magnetic permeability, so we massaged it accordingly. The
mantle was a layer of Olivine\cite{gunn}\cite{iwa}; again, our model
diverged with so low a value, so we set it to $0.3$.

The current around the boundary was zero and $\mu_{r}$ was one. Magnetic
fields were not calculated at the boundary; this gave $A($boundary$) =
\vec{0}$ as a boundary condition, consistent with space.

A Monte Carlo algorithm was used to update the points. We didn't notice
a difference at 32 grid granularity.

\section{Results}

The $\mu_{r}$ values that were used in the programming were 0.3 for Olivine,
0.9 for iron and magma, and 4 for plasma. These were determined to produce a
convergence in the Gauss-Seidel algorithm. The graphics engine assigned
a value of $1 - \mu$ to the colour and the size of the point.

\begin{table}[!htb]
\begin{center}
\begin{tabular}{r c c c}
 & Error & Frame & Zero \\
\hline
Space &         0.00 & 0 &  0 \\
Planet Core &  36.15 & 1 & 58 \\
Planet Crust &  0.69 & 0 & 50 \\
Death Star &    0.00 & 0 &  0 \\
Poof &       9486.97 &27 &117
\end{tabular}
\caption{Normalised maximum average errors, the frame that it happens,
and the frame that the errors to go to zero. The Death Star was too
far away from the planet to have an effect on it's magnetic field.}
\end{center}
\label{err}
\end{table}

As seen in Table \ref{err}, the maximum variances were made by introducing new elements into the simulation, as expected. These variances are exact because the simulation was the same every time. The `Poof' phase, values
were not static, the Earth was blowing up, so high values were expected.

\fig{death}{Emergence of Death Star.}

We see in Figure \ref{death} the station arriving, entering Earth's orbit.
Notice the magnetic lines emanating from the Earth produced by the
geodynamo.

\fig{positron}{Positron beam from Death Star.}

In Figure \ref{positron}, after Grand Moff Tarkin gives the go-ahead, the
positron beam is unleashed on the unsuspecting population of Earth.

\fig{exploded}{Planet exploding.}

In Figure \ref{exploded} we see the plasma remnants of the Earth. This
was taken just before the field created by the vaporised magma
dissipated thought space.

\section{Conclusion}
According to preliminary results, it shows that the Earth cannot
survive a positron beam attack. The magnetic field were
successfully generated from the modeled currents with adequate
precision. With modified $\mu$ values to prevent divergence, the
Gauss-Seidel iteration method proved reasonably successful.

\bibliography{tex}

\onecolumn
\appendix

\section{Simulation.c}
{\tiny \verbatimtabinput{Simulation.c}}

\section{Simulation.h}
{\tiny \verbatimtabinput{Simulation.h}}

\section{Open.c}
{\tiny \verbatimtabinput{Open.c}}

\end{document}
