\documentclass[twocolumn]{article}

% imports
\usepackage{times}      % font
\usepackage{graphicx}   % include graphics
\usepackage{fullpage}   % book margins -> std margins
\usepackage{amsmath}    % {align}
\usepackage{wrapfig}    % {wrapfigure}
\usepackage{moreverb}   % \verbatimtabinput

\bibliographystyle{plain}

% define "\fig"
\def\fig#1#2{\begin{figure}[!ht]\begin{center}
\includegraphics[width=0.24\textwidth]{#1.jpg}
\end{center}\caption{#2}\label{#1}\end{figure}}

% lists
\renewcommand{\labelenumi}{\arabic{enumi})}
\renewcommand{\labelenumii}{\alph{enumii})}
\renewcommand{\vec}[1]{\boldsymbol{#1}}

% Muhamed Razalee Yusoff -- 260263544,
% Neil Edelman -- 110121860,
% Daniel Smilowitz -- 260214511,
% Olivier Shelbaya -- 260279844

% info
\author{Muhamed Razalee Yusoff, Neil Edelman, Daniel Smilowitz, Olivier Shelbaya}
\title{CERN Destroys Alien Ship}
\date{2009-04-02}

\begin{document}\maketitle

\section{Introduction}
Antennae are devices that have become commonplace in today's world. They allow an analogue (or digital) signal carried
via wire to be transmitted through space by emitting electric fields, which then induce a current in another antenna,
further away. As a consequence of the varying electric field, a magnetic field is simultaneously generated, although
it is smaller than the electric field by a factor $\frac{1}{c}$. For high powered antennae, as employed in this simulation,
the magnetic field becomes non-negligable.
\par

\noindent This becomes very important in the practical concept demonstrated in this work: the destruction of a spaceship. Spaceships
are generally terrifying vessels of destruction which can allow for the birth of interplanetary empires of evil. It is
therefore very important to know how to destroy them, theoretically anyways...
\par

A high powered EM wave (Gigahertz range frequency) is sent via high gain antenna (i.e. most of the signal entering the antenna by wire exits as an EM 
field) into an unsuspecting spaceship, driving the spaceship to its \textbf{resonance frequency},
causing the vessel to explode. Spaceships \textit{do} present many similar points to oscillators. For one, they have matter,
and matter oscillates. Our hypothesis is therefore verified, and the math may begin.

\section{Theory}
\noindent In order to accomplish simulation, the following concepts were employed:

\noindent The poynting vector is derived from the definition of \textbf{energy} in an electromagnetic
field:

\begin{equation}
U_{em} = \frac{1}{2}\int \Bigg( \epsilon_{0}E^{2} + \frac{1}{\mu_{0}}B^{2} \Bigg) d\tau
\end{equation}

\noindent Using the above definition, along with the lorentz force law and employing vector product rules,
it is possible to find that:

\begin{equation}
\frac{dW}{dt} \propto - \frac{1}{\mu_{0}} \oint (\textbf{E} \times \textbf{B}) \cdot d\textbf{a}
\end{equation}

\noindent That is, the energy per unit time and area can be described by the \textbf{Poynting Vector}:

\begin{equation}
\vec{\textbf{S}} = \frac{1}{\mu_{0}}(\textbf{E} \times \textbf{B})
\end{equation}

\noindent As can be seen by analyzing the cross product of the definition, since E and B are perpendicular by definition, the poynting vector points 
in the direction of propagation of the EM wave.
\par
The next definition of interest is that of the \textbf{energy and momentum} for an EM wave. Although the physical quantity is small, electromagnetic 
waves nevertheless carry a small quantity of energy (obviously much less than a sound wave, for example). Equation (1) describes the energy per unit 
volume (in 3-space) of an EM wave. 

\noindent The \textbf{momentum} of an electromagnetic wave is, by definition, in terms of the speed of light, \textit{c}

\begin{equation}
\vec{\textbf{P}}_{em} = \frac{1}{c^{2}} \vec{\textbf{S}}
\end{equation}

\noindent Since we are in the presence of time-dependant potentials and fields, in order to calculate the fields at a distant point of interest as a 
function of time, \textbf{Jefimenko's Equations}, which are the electrodynamic generalisations of the coulomb and biot-savart laws were employed.
\par
The electric field Jefimenko equation is:

\tiny
\begin{equation}
 \vec{E}(\vec{r},t) = 
\frac{1}{4\pi\epsilon_0}\int\left(\frac{\rho(\vec{r'},t_r)\,\vec{R}}{R^3}+\frac{\vec{R}}{R^2c}\frac{\partial\rho(\vec{r'},t_r)}{\partial 
t} - 
\frac{1}{Rc^2}\frac{\partial \vec{J}(\vec{r'},t_r)}{\partial t}\right)d\tau'
\end{equation}
\normalsize
\par
While the Magnetic field Jefimenko equation is:

\small
\begin{equation}
\vec{B}(\vec{r},t) = \frac{\mu_0}{4\pi}\int{
	\left(\frac{\vec{J}(\vec{r'},t_r)}{R^2}+\frac{ 
	\vec{\dot{J}}(\vec{r'},t_r)}{cR}\right)\times\hat{R}\;\;d\tau'}
\end{equation}
\normalsize
\par

\noindent Taking:

\begin{equation}
\vec{J}(r',t')=\vec{J(\vec{r'})}e^{i{\omega}t}e^{-i\frac{\omega}{c}R}
\end{equation}
\par

\indent We approximate:

\begin{equation}
e^{i{\omega}t}=cos\left(\frac{\omega}{c}R\right)+isin\left(\frac{\omega}{c}R\right)
\end{equation}
\par

\noindent Finally, since all of the aformentionned quantites pertain to phenomenon which are oscillatory by construction, their instantaneous values 
are of little use for modelization. The \textbf{time average} is used, as it gives a better representation of the average value, for one cycle. The 
time average operator is defined as:

\begin{equation}
\langle \rangle = \frac{1}{T} \int dt
\end{equation}

where T is the period of oscillation, in seconds.

\section{Results}
\fig{p-02a}{Step 1 : Alien Spaceship spotted entering Earth's orbit}

\fig{p-02b}{Charged particles located in CERN start to accelerate, generating EM radiation}

\fig{p-02c}{The Intensity increases further as the particles reach maximum velocity}

\fig{p-02d}{The integrity of the Spaceship can no longer tolerate the intense Electro-magnetic radiation and EXPLODES}

\section{Conclusion}
The results achieved were not as expected. The motion of the accelerating charged particles can be modelled by a ring of current (Figure 2), thereby 
producing a Torus shaped Magnetic field. However, as can be seen in the above images, the field produced was not that of a Torus shape. Nonetheless, 
the method employed was indeed successful in detering the Alien invasion as can seen by the explosion in Figure 4. 

\onecolumn

\section{Bibliography}
"Introduction to Electrodynamics", 3rd Edition, David J. Griffiths, Prentice-Hall Inc., 1999. 

\appendix

\section{Simulation.c}
{\tiny \verbatimtabinput{Simulation.c}}

\section{Simulation.h}
{\tiny \verbatimtabinput{Simulation.h}}

\section{Open.c}
{\tiny \verbatimtabinput{Open.c}}

\end{document}
